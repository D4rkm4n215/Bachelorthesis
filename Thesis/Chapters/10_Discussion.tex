\chapter{Diskussion} % (fold)
\label{sec:diskussion}
In diesem Kapitel werden die aus dem Experiment gewonnenen Erkenntnisse mit den Ergebnissen der Literaturanalyse verglichen, um die empirischen Resultate zu überprüfen.
\newline
\noindent
Im Rahmen des Experiments wurde untersucht, wie sich die Latenzzeiten von REST- und GraphQL-APIs unter verschiedenen Anfragenkomplexitäten unterscheiden, wobei die Ergebnisse zeigten, dass REST bei Anfragen mit hoher Datenlast (100 000 Ergebnistupel) unabhängig von der zugrunde liegenden Datenbank einen deutlichen Vorteil gegenüber GraphQL aufweist. Dies deckt sich mit der Literatur, die GraphQL bei solchen Szenarien als etwa 2,5-mal langsamer beschreibt \citep{restvsgraphql}.
\newline
\noindent
Für einfache Anfragen wurde in der Literatur ein Geschwindigkeitsfaktor von 0,02 zugunsten von GraphQL genannt [23], der im Experiment jedoch mehrfach übertroffen wurde. Besonders bei parametrisierten Anfragen ohne Joins zeigte GraphQL eine bis zu 1,9-mal höhere Performance als REST. Dies wurde auch bei einem Endpunkt beobachtet, der realen Anwendungen entspricht (z. B. GET /api/persons/:pid).
\newline
\noindent
Komplexere Anfragen mehrerer Tabellen oder Nodes innerhalb einer Datenbank (z. B. GET /api/persons/:pid/projects/issues) zeigten ebenfalls einen deutlichen Vorteil für GraphQL, wobei diese Überlegenheit auch in der Literatur beschrieben wird [20]. Demnach kann GraphQL Anfragen, die eine Aggregation oder den Zugriff auf verschachtelte Daten erfordern, effizienter verarbeiten, was seine Stärke bei solchen Szenarien unterstreicht.
\newline
\noindent
Bei der Erstellung von Daten werden in der Literatur keine signifikanten Effizienzunterschiede zwischen den beiden Ansätzen festgestellt \citep{graphqlreplacerest}. Diese Beobachtung wurde im Experiment bestätigt: Der Unterschied in der Median-Latenzzeit betrug maximal 7 Millisekunden.
\newline
\noindent
Ein weiterer untersuchter Aspekt betraf die Auswirkung der Datenbank, konkret einer relationalen oder Graphdatenbank, auf die zuvor genannten APIs. Dabei zeigen die parametrisierten Anfragen, dass sich die Graph- und die relationale Datenbank bei der Abfragedauer kaum unterschieden; lediglich die Abfrage nach 100 000 Tupeln rief bei der Graphdatenbank eine deutlich erhöhte Latenz hervor, und dies unabhängig von der API. Dies weicht von der Literatur ab \citep{graphrelationaldb}; die Erwartung war, dass die Anfragedauer nahezu gleichblieb. Graphdatenbanken sind für Abfragen mit hoher Komplexität optimiert, jedoch nicht unbedingt für Bulk-Extraktionen, worüber hinaus sie potenziell mehr Overhead aufweisen und mehr Speicher- sowie CPU-Operationen benötigen.
\newline
\noindent
Bei einer Abfrage, die mehr Beziehungsevalutationen benötigt  (GET /api/persons/:pid /projects/issues), war zu erkennen, dass die Graphentheorie hier einen deutlichen Vorteil aufweist: Die GraphQL-API bearbeitete die Anfrage in Kombination mit einer Graphdatenbank drei-mal schneller als eine REST-API, die mit einer Graphdatenbank operierte. Aufseiten der relationalen Datenbank war ebenfalls ein Unterschied zu erkennen, der aber mit einem Faktor von 1,6 deutlich geringer ausfiel.
\newline
\noindent
Bei der Erstellung von Daten auf der Datenbank wurde deutlich sichtbar, dass eine Graphdatenbank erheblich länger für die Erstellung benötigt. Dies ist damit zu begründen, dass bei dieser die Tupel lediglich in der Datenbank abgelegt werden, ohne eine Beziehung zueinander herzustellen. Dies erfolgt direkt bei der Erstellung der Daten, wodurch die Graphdatenbank mehr Rechenzeit für das Erstellen und die Verbindung der Knoten benötigt.  \citep{constantinov2015running}

% chapter diskussion (end)
