\chapter{Diskussion} % (fold)
\label{sec:diskussion}
In diesem Kapitel sollen die aus dem empireischen Experiment gewonnen Erkentnisse mit den Ergebnissen der Literaturanalyse vereinigt werden, um die Ergebnisse des Experiments zu überprüfen und somit die Forschungsfragen final zu beantworten.
\newline
\noindent
Innerhalb des Experiments wurde die Erkentnis gewonnen, dass Anfragen einer REST API, unabhängig der Datenbank bei einer Menge von 100.000 Ergebnistupel einen deutlichen Vorteil gegenüber einer GraphQL API bieten. Dies entspricht der Erwartung, da bereits in der Literaturanalyse beschrieben wurde das eine GraphQL API hierbei um einen Faktor von 2.5 langsamer ist\citep{restvsgraphql}.
Bei einfacheren Anfragen wurde ein Faktor von 0,02 beschrieben \citep{migrategraphql}. Dieser wurde jedoch im Experiment meist deutlich übertroffen. Die parametrisierten Anfragen, welche ohne Joins ausgeführt wurden weißen darauf hin, dass GraphQL um den Faktor 1,9 performanter war als REST.
Bei einer Komplexen Anfrage (GET /api/persons/:pid/projects/issues) zeigt sich ebenfalls ein ein Vorteil bei der Verwendung. Diese überlegenheit von GraphQL wurde ebenfalls in der Analyse beschrieben. \citep{graphqlreplacerest}

% chapter diskussion (end)
