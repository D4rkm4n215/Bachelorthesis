\chapter{Grundlagen} % (fold)
\label{sec:grundlagen}
Die folgenden Abschnitte sollen die theoretischen Grundlagen vermitteln, die notwendig sind, um das Thema dieser Thesis zu betrachten. Die Konzepte, die hier beschrieben werden sind APIs, wie REST und GraphQL, als auch relationale und Graph Datenbanken.
\section{Relationale Algebra} % (fold)
\label{sec:relationaleAlgebra}
% section relationaleAlgebra (end)

\section{Graphentheorie} % (fold)
\label{sec:graphentheorie}
% section graphentheorie (end)

\section{API} % (fold)
\label{sec:apigrundlagen}
Nachfolgend werden die Grundlagen von APIs thematisiert. Hierbei werden die grundlegenden Definitionen im Zusammenhang mit APIs und die verschiedenen Typen vorgestellt.
\subsection{Definition API} % (fold)
\label{sec:grundlegendedefinitionvonapi}
Der Begriff \glqq API\grqq{}  steht für \glqq Application Programming Interface\grqq{}. Eine API bezeichnet eine Schnittstelle, welche Entwicklern den Zugriff auf Daten und Informationen ermöglicht. Bekannte Beispiele für häufig genutzte APIs sind die Twitter- und Facebook-APIs. Diese sind für Entwickler zugänglich und ermöglichen die Interaktion mit der Software von Twitter und Facebook. Zudem ermöglichen APIs die Kommunikation zwischen Anwendungen. Sie bieten den Anwendungen einen Weg, miteinander über das Netzwerk, überwiegend das Internet, in einer gemeinsamen Sprache zu kommunizieren. \citep{apistrategyguide}
%subsection grundlegendedefinitionvonapi (end)
\subsection{REST API} % (fold)
\label{sec:restapi}
 \textbf{Representational State Transfer (REST)} wurde erstmals im Jahr 2000 in einer Dissertation von Roy Fielding beschrieben. Hierbei handelt es sich um einen Software-Architekturstil für APIs. REST basiert auf einer Ressourcenorientierung, bei der jede Entität als Ressource betrachtet und durch eine eindeutige Uniform Resource Locator (URL) identifiziert wird. Die Architektur basiert auf sechs grundlegenden Beschränkungen, darunter die Client-Server-Architektur, bei der Client und Server unabhängig voneinander agieren. Ein wesentliches Charakteristikum von REST ist die Zustandslosigkeit, d. h. jede Anfrage beinhaltet sämtliche für die Verarbeitung erforderlichen Informationen, wodurch die Interaktion zwischen Client und Server vereinfacht wird. Die Umsetzung der CRUD-Operationen (Create, Read, Update, Delete) erfolgt durch die HTTP-Methoden (POST, GET, PUT, DELETE). REST nutzt das in HTTP integrierte Caching, um die Antwortzeiten und die Leistung zu optimieren. Dabei besteht die Möglichkeit, Serverantworten als cachefähig oder nicht cachefähig zu kennzeichnen. Des Weiteren ist eine einheitliche Schnittstelle zu nennen, welche die Interaktionen zwischen unterschiedlichen Geräten und Anwendungen erleichtert und sichtbar macht. Darüber hinaus erfordert REST ein mehrschichtiges System, bei dem jede Komponente lediglich mit der unmittelbar vorgelagerten Schicht interagiert. Die Bereitstellung von ausführbarem Code durch den Server ist optional. RESTful APIs, die diesen Prinzipien folgen, nutzen HTTP-Anfragen, um Ressourcen effizient zu bearbeiten. \citep{Fielding2000}  \citep{graphqlreplacerest}
%subsection restapi (end)
\subsection{GraphQL} % (fold)
\label{sec:graphql}
\textbf{GraphQL} wurde 2012 von Facebook für den internen Gebrauch entwickelt. Im Jahr 2015 erfolgte die Veröffentlichung als Open-Source-Projekt für die Allgemeinheit. Das Kernkonzept von GraphQL basiert auf client-getriebenen Abfragen, bei denen der Client die Struktur der Daten präzise definiert und nur die erforderlichen Daten erhält. Dies resultiert in einer Reduktion von Datenübertragungen und ermöglicht effizientere Netzwerkaufrufe. Die hierarchische Struktur der Abfragen, welche die Graph-Struktur widerspiegelt, erlaubt eine intuitive Datenmodellierung. Die starke Typisierung in GraphQL wird durch ein Schema definiert, welches die Typen der Daten spezifiziert. Dadurch wird eine bessere Validierung und Dokumentation ermöglicht. Im Gegensatz zu REST, bei dem für verschiedene Operationen mehrere Endpunkte erforderlich sind, verwendet GraphQL lediglich einen einzigen Endpunkt für alle API-Abfragen.  \citep{graphqlreplacerest}
%subsection graphql (end)
% section apigrundlagen (end)

\section{Datenbank} % (fold)
\label{sec:datenbankGrundlagen}
Im Folgenden werden die Grundlagen von Datenbanken behandelt. Es werden grundlegende Definitionen im Zusammenhang mit Datenbanken und die verschiedenen Arten von Datenbanken vorgestellt.
\subsection{Definition Datenbank und Datenbank Management System} % (fold)
\label{sec:definitiondatenbank}
Eine Datenbank stellt eine Sammlung von Daten und Informationen dar, welche für einen einfachen Zugriff gespeichert und organisiert werden. Dies umfasst sowohl die Verwaltung als auch die Aktualisierung der Daten. Die in der Datenbank gespeicherten Daten können nach Bedarf hinzugefügt, gelöscht oder geändert werden. Die Funktionsweise von Datenbanksystemen basiert auf der Abfrage von Informationen oder Daten, woraufhin entsprechende Anwendungen ausgeführt werden. DBMS bezeichnet eine Systemsoftware, die für die Erstellung und Verwaltung von Datenbanken eingesetzt wird. Zu den Funktionalitäten zählen die Erstellung von Berichten, die Kontrolle von Lese- und Schreibvorgängen sowie die Durchführung einer Nutzungsanalyse. Das DBMS fungiert als Schnittstelle zwischen den Endnutzern und der Datenbank, um die Organisation und Manipulation von Daten zu erleichtern. Die Kernfunktionen des DBMS umfassen die Verwaltung von Daten, des Datenbankschemas, welches die logische Struktur der Datenbank definiert, sowie der Datenbank-Engine, welche das Abrufen, Aktualisieren und Sperren von Daten ermöglicht. Diese drei wesentlichen Elemente dienen der Bereitstellung standardisierter Verwaltungsverfahren, der Gleichzeitigkeit, der Wiederherstellung, der Sicherheit und der Datenintegrität. \citep{9677042}

%subsubsection definitiondatenbank (end)

\subsection{Relationale Datenbank} % (fold)
\label{sec:relationaleDatenbanken}
Relationale Datenbanken basieren auf dem von E. F. Codd eingeführten relationalen Modell. Es verwendet relationale Algebra und Tupel-Relationen und speichert Daten in tabellarischer Form, wobei Zeilen als Tupel und Spalten als Attribute bezeichent werden. Dies hat zur Folge, dass die Struktur, in der die Daten gespeichert werden sollen vor der Speicherung in der Datenbank bekannt sein müssen. Falls Werte nicht vorkammen werden diese auf null gesetzt. Die Tabellen sind durch Primär- und Fremdschlüssel miteinander verknüpft. Diese Art von Datenbanken sind einfach zu entwerfen und umzusetzen und zeichnen sich durch Benutzerfreundlichkeit, Konsistenz und Flexibilität aus. Diese Datenbanken eignen sich besonders für normalisierte Daten und solche, die Transaktionsintegrität erfordern.
 \citep{relationalDatabase}  \citep{9677042}
%subsection relationaleDatenbanken (end)
\subsection{Graphdatenbanken} % (fold)
\label{sec:graphDatenbanken}
Graphdatenbanken basieren auf dem Konzept der Graphentheorie, die Daten in Form von Graphen schemafreien Weise speichern. Eine Graphdatenbank ist eine Sammlung von Knoten und Kanten, wobei die Knoten Entitäten oder Objekte darstellen und die Kanten die Beziehungen zwischen den Knoten darstellen. Der Graph  enthält auch Informationen über die Eigenschaften der mit den Knoten verbundenen Objekte. Graphdatenbanken bieten eine effiziente Datenspeicherung speziell für semistrukturierte Daten. Die Formulierung von Abfragen als Traversale in Graphen-Datenbanken sind sie schneller als relationale Datenbanken. Graphdatenbanken befolgen ACID-Bedingungen und bieten Rollback-Unterstützung, die die Konsistenz der Informationen Informationen garantiert.
\citep{9677042}
%subsection graphDatenbanken (end)
% section datenbankGrundlagen (end)

% chapter grundlagen (end)