\chapter{Einleitung} % (fold)
\pagenumbering{arabic}
\label{sec:einleitung}
\section{Motivation} % (fold)
\label{sec:motivation}
In der modernen Softwareentwicklung spielen APIs (Application Programming Interfaces) eine entscheidende Rolle bei der Integration und Kommunikation zwischen verschiedenen Diensten und Anwendungen. Traditionell wurde REST (Representational State Transfer) als Standard für die Erstellung und Nutzung von APIs verwendet. Mit der Einführung und zunehmenden Verbreitung von GraphQL, einer Abfragesprache für APIs, die von Facebook entwickelt wurde, stehen Entwickler nun vor der Wahl zwischen diesen beiden Ansätzen.
Zusätzlich gewinnt die Wahl der zugrunde liegenden Datenbanktechnologie an Bedeutung, da sie maßgeblich beeinflusst, wie effektiv REST und GraphQL implementiert werden können. Relationale Datenbanken, die auf strukturierten Tabellen und SQL basieren, bieten bewährte Mechanismen für komplexe Abfragen und garantieren hohe Datenintegrität. In Kombination mit REST ermöglichen sie eine klare Strukturierung von Endpunkten und eine stabile, vorhersehbare Datenabfrage. In GraphQL hingegen können relationale Datenbanken durch Resolver genutzt werden, um gezielt nur die angeforderten Daten bereitzustellen, was die Abfrageleistung bei komplexen Datenmodellen verbessern kann.
Graphdatenbanken bieten hingegen eine natürliche Integration für stark vernetzte Daten. Sie zeigen ihre Stärken besonders in Kombination mit GraphQL, da die flexible Abfragesprache direkt auf die Eigenschaften von Graphdatenbanken abgestimmt ist und tiefe, verknüpfte Abfragen effizient ermöglicht. Im Kontext von REST hingegen können Graphdatenbanken ebenfalls verwendet werden, erfordern jedoch oft eine zusätzliche Ebene der Verarbeitung, um die Netzwerkstruktur in flache, hierarchische API-Endpunkte zu überführen. Die Wahl zwischen REST und GraphQL hat signifikante Auswirkungen auf die Entwicklung und den Betrieb der Anwendung, insbesondere im Zusammenspiel mit der zugrunde liegenden Datenbanktechnologie. Unternehmen müssen eine fundierte Entscheidung treffen, welche Kombination aus API-Architektur und Datenbank besser zu ihren Anforderungen im Hinblick auf Leistungsfähigkeit, Skalierbarkeit und Anpassungsfähigkeit passt.
% section motivation (end)
\section{Forschungsfragen} % (fold)
\label{sec:forschungsfragen}
Nachfolgend sollen die Forschungsfragen vorgestellt werden, die aus der Motivation abgeleitet wurden. Diese dienen als Grundlage der Forschung für diese Thesis.
\newpage
\begin{itemize}
	\item \textbf{FF-1: Wie unterscheiden sich GraphQL und REST hinsichtlich der Latenzzeit bei unterschiedlichen Anfragenkomplexitäten?}  Diese Frage zielt darauf ab, die Performance beider Systeme unter variablen Bedingungen zu vergleichen. Beispielsweise könnte untersucht werden, wie schnell eine API auf eine einfache Datenabfrage reagiert, im Vergleich zu einer komplexeren, die mehrere Abhängigkeiten involviert. Diese Untersuchung könnte Einblicke in die Effizienz der beiden Technologien bieten und somit als Entscheidungshilfe für Entwickler dienen, die die beste Lösung für ihre spezifischen Bedürfnisse auswählen möchten.
	\item \textbf{FF-2: Wie beeinflussen graph- und relationale Datenbanken die Latenz von REST- und GraphQL-APIs?} Diese Frage zielt darauf ab, den Einfluss der zugrundeliegenden Datenbanktechnologien auf die Latenzzeiten von API-Anfragen zu untersuchen. Dabei wird speziell betrachtet, wie sich die Wahl einer graphbasierten Datenbank im Vergleich zu einer relationalen Datenbank auf die Antwortzeiten der APIs auswirkt. Ziel ist es, herauszufinden, wie verschiedene Datenbankmodelle die Effizienz der API-Interaktionen beeinflussen und welche Datenbanktechnologie die besten Latenzwerte für unterschiedliche Anwendungsfälle bietet.
\end{itemize}
% section forschungsfragen (end)
\section{Vorgehensweise} % (fold)
\label{sec:vorgehensweise}
Die Untersuchung basiert auf einer Kombination aus theoretischen Analysen und empirischen Experimenten, um die beiden Forschungsfragen zu beantworten. Zunächst erfolgt eine umfassende Literaturrecherche, die bestehende Studien zu den Performance-Unterschieden zwischen GraphQL und REST sowie die Auswirkungen von graph- und relationalen Datenbanken auf die Latenz von API-Anfragen behandelt. Ziel ist es, ein fundiertes Verständnis der Performance-Differenzen beider Technologien zu entwickeln und herauszufinden, wie unterschiedliche Datenbanktechnologien die Antwortzeiten beeinflussen.
In den empirischen Experimenten werden APIs sowohl mit REST als auch mit GraphQL unter Verwendung von graph- und relationalen Datenbanken implementiert. Dabei werden Performance-Tests durchgeführt, um die Anfrage- und Antwortzeiten unter verschiedenen Lastbedingungen und Anfragenkomplexitäten zu messen. Ein besonderes Augenmerk liegt auf der Analyse der Latenzzeiten, sowohl bei einfachen als auch bei komplexeren Abfragen, die mehrere Abhängigkeiten beinhalten. Ziel der empirischen Untersuchung ist es, herauszufinden, wie die Wahl der Datenbanktechnologie die Latenz der API beeinflusst und welche Kombination aus API-Technologie und Datenbank für unterschiedliche Anwendungsfälle die besten Performance-Werte bietet. Diese Ergebnisse können Entwicklern als Entscheidungshilfe dienen, um die geeignetste Technologie für ihre spezifischen Anforderungen auszuwählen.
% section vorgehensweise (end)
% chapter einleitung (end)