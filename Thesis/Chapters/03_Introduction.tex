\chapter{Einleitung} % (fold)
\pagenumbering{arabic}
\label{sec:einleitung}
\section{Motivation} % (fold)
\label{sec:motivation}
In der modernen Softwareentwicklung sind APIs (Application Programming Interfaces) zentral für die Verbindung und den Austausch zwischen verschiedenen Diensten und Anwendungen. Traditionell war REST (Representational State Transfer) die bevorzugte Methode, um APIs zu erstellen. Mit der Einführung von GraphQL, einer von Facebook entwickelten Abfragesprache, haben Entwickler jedoch eine weitere Option, die zunehmend an Bedeutung gewinnt.

\noindent
Die Wahl der passenden API-Architektur hängt eng mit der eingesetzten Datenbanktechnologie zusammen, da diese die Effizienz und Flexibilität der Implementierung stark beeinflusst. Relationale Datenbanken, die auf strukturierten Tabellen basieren, gelten als etabliert und ermöglichen präzise Datenabfragen mit hoher Zuverlässigkeit. In Kombination mit REST bieten sie eine klare und stabile Struktur für den Zugriff auf Daten. GraphQL hingegen bietet die Möglichkeit, gezielt nur die benötigten Daten abzufragen, was besonders bei komplexen Modellen vorteilhaft sein kann.

\noindent
Für Anwendungen, die stark vernetzte Daten verarbeiten, bieten Graphdatenbanken eine natürliche Grundlage. In Verbindung mit GraphQL lassen sich komplexe Abfragen effizient umsetzen, da die Technologie gut auf die Eigenschaften solcher Datenbanken abgestimmt ist. Auch mit REST können Graphdatenbanken genutzt werden, allerdings ist hier oft zusätzlicher Aufwand nötig, um die Daten in eine geeignete Form zu bringen.

\noindent
Die Entscheidung zwischen REST und GraphQL sowie der passenden Datenbanktechnologie hat weitreichende Konsequenzen für die Entwicklung und den Betrieb einer Anwendung. Unternehmen sollten sorgfältig prüfen, welche Kombination aus API-Ansatz und Datenbank ihren Anforderungen am besten entspricht – sei es in Bezug auf Leistung, Skalierbarkeit oder Flexibilität.
% section motivation (end)
\newpage
\section{Forschungsfragen} % (fold)
\label{sec:forschungsfragen}
Nachfolgend sollen die Forschungsfragen vorgestellt werden, die aus der Motivation abgeleitet wurden. Diese dienen als Grundlage der Forschung für diese Arbeit.
\begin{itemize}
	\item \textbf{FF-1: Wie unterscheiden sich GraphQL und REST hinsichtlich der Latenzzeit bei unterschiedlichen Anfragenkomplexitäten ?}  Diese Frage zielt darauf ab, die Performance beider Systeme unter variablen Bedingungen zu vergleichen. Beispielsweise könnte untersucht werden, wie schnell eine API auf eine einfache Datenabfrage reagiert, im Vergleich zu einer komplexeren, die mehrere Abhängigkeiten involviert. Diese Untersuchung könnte Einblicke in die Effizienz der beiden Technologien bieten und somit als Entscheidungshilfe für Entwickler dienen, die die beste Lösung für ihre spezifischen Bedürfnisse auswählen möchten.
	\item \textbf{FF-2: Wie beeinflussen graph- und relationale Datenbanken die Latenz von REST- und GraphQL-APIs ?} Hierbei soll der Einfluss der zugrundeliegenden Datenbanktechnologien auf die Latenzzeiten von API-Anfragen untersucht werden. Dabei wird speziell betrachtet, wie sich die Wahl einer graphbasierten Datenbank im Vergleich zu einer relationalen Datenbank auf die Antwortzeiten der APIs auswirkt. Ziel ist es, herauszufinden, wie verschiedene Datenbankmodelle die Effizienz der API-Interaktionen beeinflussen und welche Datenbanktechnologie die besten Latenzwerte für unterschiedliche Anwendungsfälle bietet.
\end{itemize}
% section forschungsfragen (end)
\section{Ziel der Arbeit} % (fold)
\label{sec:zielderarbeit}
Das Ziel dieser Arbeit ist es, die Leistungsfähigkeit und Effizienz von REST- und GraphQL-APIs im Zusammenspiel mit relationalen und graphbasierten Datenbanken zu untersuchen und miteinander zu vergleichen. Im Mittelpunkt steht dabei die Analyse, wie sich die Wahl der API-Architektur und der zugrunde liegenden Datenbanktechnologie auf die Latenzzeiten und die Effizienz bei Anfragen unterschiedlicher Komplexität auswirken. Es soll aufgezeigt werden, welche Wechselwirkungen zwischen API-Architektur und Datenbanktechnologie bestehen und wie diese die Leistungsfähigkeit von modernen API-Systemen beeinflussen. Mithilfe der definierten Forschungsfragen wird eine Analyse durchgeführt, die praxisrelevante Einsichten für die Implementierung effizienter API-Systeme liefert.
% section zielderarbeit (end)
\newpage
\section{Vorgehensweise} % (fold)
\label{sec:vorgehensweise}
Die Untersuchung basiert auf einer Kombination aus theoretischen Analysen und empirischen Experimenten, um die beiden Forschungsfragen zu beantworten. Zunächst erfolgt eine umfassende Literaturrecherche, die bestehende Studien zu den Performance-Unterschieden zwischen GraphQL und REST sowie die Auswirkungen von graph- und relationalen Datenbanken auf die Latenz von API-Anfragen behandelt. Ziel ist es, ein fundiertes Verständnis der Performance-Differenzen beider Technologien zu entwickeln und herauszufinden, wie unterschiedliche Datenbanktechnologien die Antwortzeiten beeinflussen.
In den empirischen Experimenten werden APIs sowohl mit REST als auch mit GraphQL unter Verwendung von graph- und relationalen Datenbanken implementiert. Dabei werden Performance-Tests durchgeführt, um die Anfrage- und Antwortzeiten bei verschiedenen Anfragenkomplexitäten zu messen. Ziel der empirischen Untersuchung ist es, herauszufinden, wie die Wahl der Datenbanktechnologie die Latenz der API beeinflusst und welche Kombination aus API-Technologie und Datenbank für unterschiedliche Anwendungsfälle die besten Performance-Werte bietet. Diese Ergebnisse können praxisrelevante Erkentnisse liefern, um die geeignetste Technologie für spezifischen Anforderungen auszuwählen.
% section vorgehensweise (end)
% chapter einleitung (end)