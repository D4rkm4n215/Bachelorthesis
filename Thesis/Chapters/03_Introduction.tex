\chapter{Einleitung} % (fold)
\pagenumbering{arabic}
\label{sec:einleitung}
\section{Motivation} % (fold)
\label{sec:motivation}
In der modernen Softwareentwicklung sind Application Programming Interfaces (APIs) zentral für die Verbindung und den Austausch zwischen verschiedenen Diensten und Anwendungen. Traditionell stellt der Representational State Transfer (REST) die bevorzugte Methode zur Erstellung von APIs dar, doch mit der Einführung von GraphQL, einer von Facebook entwickelten Abfragesprache, bietet sich Entwicklern eine weitere Option, die zunehmend an Bedeutung gewinnt.

\noindent
Die Wahl der passenden API-Architektur hängt eng mit der eingesetzten Datenbanktechnologie zusammen, weil diese die Effizienz und Flexibilität der Anwendung stark beeinflusst. Relationale Datenbanken, die auf strukturierten Tabellen basieren, gelten als etabliert und ermöglichen präzise Datenabfragen mit hoher Zuverlässigkeit, wobei sie in Kombination mit REST eine klare und stabile Struktur für den Zugriff auf Daten bieten. GraphQL hingegen erlaubt die gezielte Abfrage der benötigten Daten, was besonders bei komplexen Modellen vorteilhaft sein kann.

\noindent
Für Anwendungen, die intensiv mit vernetzten Daten arbeiten, stellen Graphdatenbanken eine ideale Basis dar. In Verbindung mit GraphQL lassen sich komplexe Abfragen effizient umsetzen, weil die Technologie auf die Eigenschaften solcher Datenbanken abgestimmt ist. Auch mit REST können Graphdatenbanken genutzt werden, allerdings ist hier oft zusätzlicher Aufwand nötig, um die Daten in eine geeignete Form zu bringen.

\noindent
Die Entscheidung zwischen REST und GraphQL sowie für die passende Datenbanktechnologie hat weitreichende Konsequenzen für die Entwicklung und den Betrieb einer Anwendung. Unternehmen sollten daher genau prüfen, welche Kombination aus API-Ansatz und Datenbank ihren Anforderungen entspricht, sei es in Bezug auf Leistung, Skalierbarkeit oder Flexibilität.
% section motivation (end)
\newpage
\section{Forschungsfragen} % (fold)
\label{sec:forschungsfragen}
Nachfolgend werden die Forschungsfragen vorgestellt, die sich aus der Motivation ableiten und als Grundlage der Forschung dieser Arbeit dienen.
\newline
\noindent
Die übergeordnete Forschungsfrage lautet: Wie wirkt sich die Wahl der API-Architektur und der zugrunde liegenden Datenbanktechnologie auf die Latenzzeiten bei Anfragen unterschiedlicher Komplexität aus?
\newline
Diese wurde in die nachfolgenden Forschungsfragen FF1 und FF2 unterteilt, die gesondert beantwortet werden und somit die übergeordnete Forschungsfrage beantworten.
\begin{itemize}
	\item \textbf{FF1: Wie unterscheiden sich GraphQL und REST hinsichtlich der Latenzzeit bei unterschiedlichen Anfragenkomplexitäten?}  Diese Frage zielt darauf ab, die Performance beider Systeme unter variablen Bedingungen zu vergleichen. Beispielsweise wird untersucht, wie schnell eine API auf eine einfache Datenabfrage reagiert, im Vergleich zu einer komplexeren Abfrage, die mehrere Abhängigkeiten involviert. Diese Ermittlung soll Einblicke in die Effizienz der beiden Technologien bieten und somit als Entscheidungshilfe für Entwickler dienen, um die optimale Lösung für ihre spezifischen Bedürfnisse auszuwählen.
	\item \textbf{FF2: Wie beeinflussen Graph- und relationale Datenbanken die Latenz von REST- und GraphQL-APIs?} Hierbei wird der Einfluss der zugrunde liegenden Datenbanktechnologien auf die Latenzzeiten von API-Anfragen untersucht, wobei speziell betrachtet wird, wie sich die Wahl einer graphbasierten Datenbank im Vergleich zu einer relationalen Datenbank auf die Antwortzeiten der APIs auswirkt. Ziel ist es, herauszufinden, wie verschiedene Datenbankmodelle die Effizienz der API-Interaktionen beeinflussen und welche Datenbanktechnologie die niedrigsten Latenzwerte für unterschiedliche Anwendungsfälle bietet.
\end{itemize} 
% section forschungsfragen (end)
\section{Ziel der Arbeit} % (fold)
\label{sec:zielderarbeit}
Das Ziel dieser Arbeit besteht darin, die Leistungsfähigkeit und Effizienz von REST- sowie GraphQL-APIs im Zusammenspiel mit relationalen und graphbasierten Datenbanken zu untersuchen sowie miteinander zu vergleichen. Im Mittelpunkt steht dabei die Analyse, wie sich die Wahl der API-Architektur und der zugrunde liegenden Datenbanktechnologie auf die Latenzzeiten und die Effizienz bei Anfragen unterschiedlicher Komplexität auswirkt. Es soll aufgezeigt werden, welche Wechselwirkungen zwischen API-Architektur sowie Datenbanktechnologie bestehen und wie diese die Leistungsfähigkeit moderner API-Systeme beeinflussen. Auf Basis der definierten Forschungsfragen wird eine Analyse durchgeführt, die praxisrelevante Einsichten für die Implementierung effizienter API-Systeme liefert.
% section zielderarbeit (end)
\newpage
\section{Vorgehensweise} % (fold)
\label{sec:vorgehensweise}
Das Ziel dieser Arbeit besteht darin, die Leistungsfähigkeit und Effizienz von REST- sowie GraphQL-APIs im Zusammenspiel mit relationalen und graphbasierten Datenbanken zu untersuchen sowie miteinander zu vergleichen. Es soll aufgezeigt werden, welche Wechselwirkungen zwischen API-Architektur sowie Datenbanktechnologie bestehen und wie diese die Leistungsfähigkeit moderner API-Systeme beeinflussen. Auf Basis der definierten Forschungsfragen wird eine Analyse durchgeführt, die mithilfe von Studien und anderen Arbeiten deren Beantwortung unterstützt. Außerdem wird ein umfassendes Experiment durchgeführt, bei dem die API-Typen mit den Datenbankentechnologien implementiert und auf ihre Latenz bei unterschiedlichen Anfragen getestet werden. Im Mittelpunkt steht dabei die Analyse, wie sich die Wahl der API-Architektur und der zugrunde liegenden Datenbanktechnologie auf die Latenzzeiten und die Effizienz bei Anfragen unterschiedlicher Komplexität auswirkt. 
% section vorgehensweise (end)
% chapter einleitung (end)