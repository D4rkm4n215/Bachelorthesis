\chapter{Ausblick} % (fold)
\label{sec:ausblick}
Im Rahmen dieser Arbeit wurden GraphQL und REST in Kombination mit relationalen sowie Graphdatenbanken analysiert, wobei zahlreiche Ansatzpunkte für weiterführende Untersuchungen identifiziert wurden, die sowohl technische als auch konzeptionelle Aspekte betreffen.

\noindent
Ein erstes Potenzial für zukünftige Arbeiten liegt in der Betrachtung der Implementierung von APIs unter Verwendung unterschiedlicher Programmiersprachen. Die in dieser Arbeit analysierten Implementierungen konzentrierten sich auf spezifische Technologien, was die Frage offenlässt, ob andere Programmiersprachen, insbesondere solche mit unterschiedlichen Paradigmen (z. B. funktional oder objektorientiert), ebenfalls einen Einfluss auf die Latenz oder andere Leistungsparameter der APIs haben. Eine vergleichende Analyse könnte hierzu wertvolle Erkenntnisse liefern.

\noindent
Im Bereich der Datenbanken wäre es ebenfalls lohnend, die Untersuchung auf weitere Datenbanktypen auszudehnen. Besonders dokumentenbasierte NoSQL-Datenbanken wie MongoDB oder Couchbase bieten Potenzial für ergänzende Analysen, weil sie aufgrund ihrer strukturellen Unterschiede gegenüber relationalen und graphbasierten Datenbanken relevante Alternativen darstellen. Solche Untersuchungen könnten Aufschluss darüber geben, inwiefern dokumentenbasierte Datenbanken spezifische Vorteile oder Herausforderungen für den Einsatz mit REST oder GraphQL bieten.

\noindent
Darüber hinaus könnte der Einsatz eines komplexeren Datenmodells bedeutende Erkenntnisse liefern. In dieser Arbeit wurde ein moderates Datenmodell genutzt, das nur eine begrenzte Anzahl an Relationen aufwies. Eine Erweiterung des Modells, beispielsweise durch eine höhere Anzahl von Tabellen oder Knoten sowie komplexere Verknüpfungen zwischen diesen, würde die Belastung der Datenbank und der API erhöhen. Solche Szenarien könnten dazu beitragen, die Skalierbarkeit und Effizienz der untersuchten Technologien unter realistischeren Bedingungen zu bewerten.

\noindent
Zusammenfassend bietet diese Arbeit eine Grundlage für weiterführende Analysen und stellt mehrere Ansätze für zukünftige Forschungen bereit, die sowohl auf theoretischer als auch auf praktischer Ebene die Relevanz und Einsatzmöglichkeiten von GraphQL sowie REST vertiefen könnten.

% chapter ausblick (end)