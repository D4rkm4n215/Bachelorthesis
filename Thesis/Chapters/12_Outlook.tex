\chapter{Ausblick} % (fold)
\label{sec:ausblick}
Im Rahmen dieser Arbeit wurden GraphQL und REST in Kombination mit relationalen sowie Graphdatenbanken analysiert. Dabei konnten zahlreiche Ansatzpunkte für weiterführende Untersuchungen identifiziert werden, die sowohl technische als auch konzeptionelle Aspekte betreffen.

\noindent
Ein erster Ansatzpunkt für zukünftige Arbeiten liegt in der Betrachtung der Implementierung von APIs unter Verwendung verschiedener Programmiersprachen. Die in dieser Arbeit analysierten Implementierungen konzentrierten sich auf spezifische Technologien, was die Frage offen lässt, ob andere Programmiersprachen, insbesondere solche mit unterschiedlichen Paradigmen (z. B. funktional oder objektorientiert), ebenfalls einen Einfluss auf die Latenz oder andere Leistungsparameter der APIs haben könnten. Eine vergleichende Analyse könnte hierbei wertvolle Erkenntnisse liefern.

\noindent
Im Bereich der Datenbanken wäre es ebenfalls lohnenswert, die Untersuchung auf weitere Datenbanktypen auszudehnen. Besonders dokumentenbasierte NoSQL-Datenbanken wie MongoDB oder Couchbase bieten Potenzial für ergänzende Analysen, da diese aufgrund ihrer strukturellen Unterschiede gegenüber relationalen und graphbasierten Datenbanken interessante Alternativen darstellen könnten. Solche Untersuchungen könnten Aufschluss darüber geben, inwiefern dokumentenbasierte Datenbanken spezifische Vorteile oder Herausforderungen für den Einsatz mit REST oder GraphQL bieten.

\noindent
Darüber hinaus könnte der Einsatz eines komplexeren Datenmodells Erkenntnisse liefern. In dieser Arbeit wurde ein eher moderates Datenmodell genutzt, das nur eine begrenzte Anzahl an Relationen aufwies. Eine Erweiterung des Modells, beispielsweise durch eine höhere Anzahl von Tabellen oder Knoten sowie komplexere Verknüpfungen zwischen diesen, würde die Belastung auf die Datenbank und die API erhöhen. Solche Szenarien könnten dazu beitragen, die Skalierbarkeit und Effizienz der untersuchten Technologien unter realistischeren Bedingungen zu bewerten.

\noindent
Zusammenfassend bietet diese Arbeit eine solide Grundlage für weiterführende Analysen und stellt mehrere Ansätze für zukünftige Forschungsarbeiten bereit, die sowohl auf theoretischer als auch auf praktischer Ebene die Relevanz und Einsatzmöglichkeiten von GraphQL und REST vertiefen könnten.

% chapter ausblick (end)