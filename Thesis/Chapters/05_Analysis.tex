\chapter{Analyse} % (fold)
\label{sec:analyse}
In diesem Kapitel sollen die in der Einleitung definierten Forschungsfragen untersucht werden. Um diese Fragen zu beantworten, wird eine Literaturanalyse durchgeführt, die bestehende Studien, Bücher und wissenschaftliche Artikel betrachtet. Ziel ist es, aus der vorhandenen Literatur systematisch Erkenntnisse abzuleiten, die die Performance der Ansätze in verschiedenen Szenarien beleuchten.

\section{FF-1: Wie unterscheiden sich GraphQL und REST hinsichtlich der Latenzzeit bei unterschiedlichen Anfragenkomplexitäten?} % (fold)
\label{sec:ff1}
Um zu untersuchen, wie GraphQL und REST sich im Bezug auf Latenz bei verschiedenen Anfragekomplexitäten auswirkt, muss man zunächst darauf eingehen, welche Faktoren oder Spezifikationen, die die APIS nutzen sich auf die Latenz auswirken.
Bei der Bereitstellung von Daten setzt REST auf HTTP-Endpunkte. Diese untestützen nativ HTTP-Caching,  wodurch Ressourcen in einem Cache zwischengespeichert werden können um unnötige Datenübertragungen und Serveranfragen zu vermeiden und somit die Zugriffszeiten zu verringern.
Bei GraphQL wird dies nativ nicht unterstützt. Hierdurch können wiederkehrende Anfragen nicht gecached werden und müssen jeweils immer vom Server bearbeitet werden, wodurch eine höhere Zugriffszeit entsteht. \citep{graphqlreplacerest}

\noindent
REST fördert die Zerlegung von Systemen in eine Menge verknüpfter Ressourcen mit einem bestimmten Granularitätsgrad. Dies führt zu schwierigen Abwägungen zwischen Wiederverwendbarkeit und Leistung, die in der allgemeinen Software-Service-Architektur wohlbekannt sind. Weniger granulare und kohäsivere Services werden bevorzugt, da sie lose Kopplung und hohe Wiederverwendbarkeit fördern. Dies kann jedoch zu komplizierten Client-Server-Interaktionen führen, bei denen mehrere aufeinanderfolgende Anfragen notwendig sind, um die benötigten Daten aus dem Ressourcengraphen abzurufen,ein Phänomen, das als „Underfetching“ bekannt ist. Dieses Problem wird auch als n+1-Problem bezeichnet und tritt bei REST auf der Seite des Clients auf. Dieser muss demensprechend weitere Anfragen schicken, bis die benötigte Antwortzeiten führen. Der gegensätzliche Ansatz des „Coarse-Grained Remote Interfaces“-Patterns reduziert die Anzahl der Anfragen und den Netzwerkaufwand, geht jedoch mit einer geringeren Kohäsion und Wiederverwendbarkeit einher.\citep{graphqlhealth} \citep{migrategraphql}

\noindent
Bei GraphQL kann es ebenso zu einem n+1 Problem kommen. Hierbei tritt dieses jedoch nicht auf der Client Seite auf sondern direkt beim Server bei der Verarbeitung der Anfrage. Der GraphQL Server muss dann mehrere Anfragen an die Datenbank schicken, um die benötigten Daten zu erhalten, um sie dann an den Client auszuliefern.
\citep{graphqlsemantics}

\noindent
Hierfür bietet GraphQL einen sogenannten Dataloader, der die Anfragen, die zur bearbeitung eines Requests benötigt werden bündelt und als eine einzelne, optimierte, Datenbankabfrage ausführt. \citep{nordstrom2022graphql}

%section ff1 (end)
\section{FF-2: Wie beeinflussen graph- und relationale Datenbanken die Latenz von REST- und GraphQL-APIs?} % (fold)
\label{sec:ff2}
%section ff2 (end)







% chapter analyse (end)